\documentclass[moyen,bringshurstpage]{classeUPD}
\graphicspath{{/home/guillaume/Logos/},
  {/home/guillaume/recherche/figures/animaux/}}


%\OnehalfSpacing % Option pour donner l'impression que le document est
% en double interligne
\begin{document}
\frontmatter
\setboolean{these}{true} % true= these, false=hdr
\colorpage % [color] rouge du I de ParIs par defaut
%\background[width=\paperwidth]{Yeti_by_Philippe_Semeria.jpg}
% exemple d'image de fond en option les options de includegraphics
\logoUpd{Logo_P7} % le logo de la fac avec le nom du fichier
\otherlogo[2]{LogoSmallR600}{cnrs.eps}
% [nb de logos: 1 ou 2]{logo1}{logo2} 
% ne fonctionne pas si nb=0
\labo[2]{Laboratoire de Moulinsart en Sciences C\^alines}{chez Ouioui}
% meme syntaxe
\title{De la sociologie du Y\'eti comme paradigme de
  protocivilisation. Comparaison avec la soci\'et\'e Inca} 
% trois lignes maximum, bug au-dela
\author{Tintin}
\university % P7 par defaut, [une autre fac] en option
\ufr[UFR de Truckologie] % par defaut : rien
\ed[\'Ecole doctorale machin] % idem ufr
\defense[14 juillet 1969] % par defaut le jour-dit
\speciality{\'ethologie} % obligatoire pour la these, vide sinon
\president{C. Haddock} % obligatoire sur la page de titre
\direct{Tchang}
\referees{T. Tournesol}{S. Lampion}{Rastapopoulos} 
% obligatoire sur la page de titre 
% 2 pour la these, 3 pour l'HDR
\othermember[4][Examinateurs][Dupont][][Dupond][][M\up{lle}
Castafiore][Censeur][C. Sponsz]
% [ne sert a rien][role][Nom][role][Nom][role][Nom][role][Nom] : 4
% maximum, tous optionnels

\makecover % pour la couverture suit la charte graphique de P7
%\maketitleincover % pour la page de titre idem a peu pres...
\maketitle


% Page des r\'esum\'es francais et anglais
\resumesummary{%
  Ce document, intitul\'e \thetitle ~et r\'edig\'e par \theauthor, est
  un essai pour r\'ealiser un canevas de m\'emoire de th\`ese ou
  pour une habilitation \`a diriger des recherches avec une mise en
  page propre \`a notre universit\'e. Cette nouvelle
  extension a ses fondements sur la classe \texttt{memoir}, et a
  profi\'e des commentaires de P. Wilson.}{%
  This report is a temptative either of thesis memoir, or of memoir
  for french silly suplementary diploma to become thesis master. This
  new class is based on  \texttt{memoir} class and on P. Wilson's comments.
}

% Dedicace
\begin{dedication}
\`A Milou, mon plus agr\'eable ami depuis qu'il est muet
\end{dedication}
% Table des matieres
\tableofcontents*
\cleardoublepage
\chapter{Remerciements}

Merci \`a tous ceux qui n'ont pas particip\'e.
\chapter{Prol\'egom\`enes}
\chapterprecishere{Voici quelques explications de l'extension
  classUPD}

\subsubsection{Version au 13/07/2010}
Ce document sert de \emph{tutorial} au style que j'ai cr\'e\'e, il
faudrait en faire une documentation, ce sera plus tard; je me suis
content\'e d'utiliser la classe {\ttfamily memoir} qui vraiment tr\`es
compl\`ete.  Quelques commandes suppl\'ementaires servent \`a en
simplifier l'utilisation.

\subsubsection{Premi\`eres pages}
J'ai realis\'e une page de couverture en utilisant la charte graphique
de la fac, et deux pages de titre au standard habituel des th\`eses et
de m\'emoire d'HDR. La charte graphique fixe la position du logo, sa
taille, la position du titre. Mais elle laisse libre l'utilisation
d'une image de fond \marginparnote{background}  ou d'un fond de couleur
\marginparnote{colorpage} dont la couleur par d\'efaut est celui du i de
ParIs. Avec un boul\'een\marginparnote{these initialis\'e \`a true ou
  false.}, on passe d'une version \emph{th\`ese}, \`a une version \emph{HDR}.

L'id\'ee est aussi d'am\'eliorer l'apparence de ces manuscript par une
jolie couverture\marginparnote{makecover}. Cependant on peut
utiliser une page de titre comme
couverture\marginparnote{maketitleincover}, ce qui est l'apparence
habituelle des th\`eses. La seconde commande de page de
titre\marginparnote{makecover} est pr\'evue pour \^etre utilis\'ee
apr\`es celle de couverture et sert \`a fournir les informations
l\'egales.

Ensuite viennent les r\'esum\'es\marginparnote{resumesummary} en
fran\c{c}ais puis en anglais, une d\'edicace, la table des mati\`eres
et les remerciements. Ce peut \^ etre aussi le lieu pour une
pr\'eface, ou tout autre section pr\'eliminaire au manuscript
lui-m\^eme\marginparnote{born\'ee par frontmatter et mainmatter.}

\subsubsection{La mise en page}

Mon parti-pris de mise en page est de suivre la logique \LaTeX~: texte
en colonne \'etroite (ici $12,25$\,cm). Mais j'ai
allong\'e la colonne, ce qui interdit les notes de bas de page. Notez
que c'est le format le plus facile pour la lecture, c'est pourquoi les
journaux sont dispos\'es sur plusieurs colonnes. En revanche j'ai
introduit de nouvelles commandes pour valoriser l'utilisation de la
marge ext\'erieure avec la possibilit\'e d'y mettre des
notes\marginparnote{marginparnote} r\'ef\'erenc\'ees (voir l'exemple
de la note~\ref{note1} \`a la~\pref{note1}), des illustrations de
marge\marginparnote{figinmargin} avec les utilitaires habituelles
des figures, par exemple l'illustration~\ref{illustration1} ayant un
titre dans la table des illustrations diff\'erents de la l\'egende
dans le corps du texte, ou celle~\ref{ill2} sans titre mais dont le
corps de la l\'egende est directement ins\'er\'e dans la table des
illustrations, ou m\^eme sans l\'egende.  Il y a les figures
flottantes habituelles, par exemple la figure~\ref{figure:fignormal},
ou des figures\marginparnote{bigfigure} de taille plus grande mais
fix\'ee, comme la fig.~\ref{bigfig}, qui utilise en plus la moiti\'e
de la marge ext\'erieure.

Mon choix \'etant arbitraire, c'est le meilleur. Mais comme il n'est
pas forc\'ement partag\'e, j'ai introduit trois autres choix qui sont
d\'efinis comme des options de cette classe. Le
premier\marginparnote{bringshurstpage} permet d'aller
encore plus loin dans l'utilisation des marges. Il provient d'une
figure anglo-saxonne de la typographie
R. Bringshurst~\cite{Bringshurst_1999_book}, qui l'a lui-m\^eme
d\'ecouvert dans un manuscript m\'edi\'eval. 

Le deuxi\`eme\marginparnote{defaultpage} est la mise en page par
d\'efaut de la classe {\texttt memoir}. La derni\`ere
option\marginparnote{widepage} permet de contenter ceux qui ne jurent
que par les pages bien pleines.
\subsubsection{Mais que font les polices~?}
J'ai fix\'e la police\marginparnote{lmodern} {\sffamily sans serif}
pour \^etre la plus proche de la police {\sffamily Gill}, qui est
certes le choix graphique de l'universit\'e, mais n'en reste pas moins
une police propri\'etaire. En revanche, cette police est tr\`es
diff\'erente en fonte grasse. J'en ai donc supprim\'e l'utilisation
dans les titres. J'ai adopt\'e une hierarchie de taille des polices en
fonction des sous-divisions du texte~:
\begin{itemize}
\item {\sffamily\HUGE partie}
\item {\sffamily\Huge chapitre}
\item {\sffamily\huge section}
\item {\sffamily\Large sous-section}
\item {\sffamily\large sous-sous-section}
\end{itemize}

Gr\^ace \`a mes premiers lecteurs, le choix de la police normale s'est
tourn\'ee vers Utopia \marginparnote{Extension fourier}. Gr\^ace \`a
ces m\^emes commentaires, j'ai corrig\'e le style pour suivre les
r\`egles typographiques fran\c{c}aises (en tous cas, celles que je
connais ou celles qui m'ont \'et\'e communiqu\'ees, merci JB).

Pour des raisons de facilit\'e (ou de fain\'eantise), j'ai limit\'e
le nombre de tailles \`a deux~: 10\,pt et 11\,pt d\'efinies comme des
options de la classe~\marginparnote{petit et moyen}.
\subsubsection{Autre}
J'ai aussi introduit un environnement de liste pour le curriculum
vitae\marginparnote{cvlist}, par exemple~:
\begin{cvlist}[\mille]
\item[1997] d\'ecouverte de \LaTeX~;

  apprentissage rapide mais superficielle pour la r\'edaction d'un rapport
  sans pr\'etention au sujet d'un stage int\'eressant effectu\'e dans une
  ambiance sympathique~;
\item[2010] r\'ealisation d'une extension pour un document
  universitaire qui res\-pectent certaines contraintes, mais j'en connais qui
  voudraient en rajouter~;

  r\'edaction d'un document avec cette extension~;
\item[2020] soutenance d'HDR~;
\end{cvlist}
fin de l'exemple.


P. Wilson fournit une commande\marginparnote{OnehalfSpacing}
permettant de feindre le doublement de l'interligne, ce qui peut aider
\`a r\'ediger les documents \`a destination de la biblioth\`eque.

La suite\marginparnote{d\'ebutant par backmatter.} du document comprend
les annexes, les tables des figures et des
illustrations\marginparnote{C'est ainsi que je nomme les figures dans
  les marges.}, liste des tableaux et bibliographie. Il faudrait
ajouter un index et un glossaire.

\subsubsection{Les chantiers}

Bien s\^ur, il y a des restrictions et des probl\`emes~:
\begin{itemize}\tightlist
\item le titre est fix\'e \`a trois lignes maximum~;
\item il y a au plus trois tutelles, l'universit\'e ayant un place
  privil\'egi\'ee puis deux autres en pieds de page de titre, et
  seulement deux~;
\item pour le jury, j'ai fix\'e la position du pr\'esident, du
  directeur (si besoin est), des rapporteurs (deux ou trois
  selon\verb+these+)~;
\item les autres membres du jury ont une d\'esignation libre, mais
  leur nombre est limit\'e \`a quatre~;
\item il manque une quatri\`eme page de couverture~;
\item et un index~;
\item et un glossaire~;
\item et toutes les erreurs que je n'ai pas vues\dots
\end{itemize}

Ce canevas est probablement imparfait. Le code pourrait notamment
\^etre rationnalis\'e. J'esp\`ere seulement qu'il pourra servir de
base \`a un format, \`a la fois alternatif de ce qui existe
mais qui porte aussi une certaine identit\'e. Enfin et surtout, il
doit simplifier la mise en page de nos documents.
\mainmatter
\part{Intime}
\chapter[Introduction][Intro]{Chapitre Introductif}
\chapterprecishere{une vraie introduction \`a suivre}
Des tableaux
\begin{table}[h!]
\centering
\begin{tabular}{|>{\bfseries}p{1cm}|>{\large}p{1cm}|>{\Large\itshape}p{1cm}|}
\toprule
a aa aaa&a aa aaa&a aa aaa\\ \hline
b bb bbb&b bb bbb&b bb bbb\\ \hline
m mm mmm&m mm mmm&m mm mmm\\ \bottomrule
\end{tabular}
\caption[Exemple de table]{Voici un exemple de tableau, pour plus
  d'informations voir le manuel de la classe \ttfamily{memoir}.}
\end{table}
\begin{table}[h!]
\centering
\begin{tabular}{|l@{$\wr$}c@{$\natural$}r|}
\hline
a aa aaa&a aa aaa&a aa aaa\\ \hline
b bb bbb&b bb bbb&b bb bbb\\ \hline
m mm mm&m mm mm&m mm mm\\ \hline
\end{tabular}
\caption[Un autre exemple de table]{Voici un autre exemple de tableau,
  pour plus d'informations voir le manuel de la classe
  {\ttfamily memoir}, m\^eme si celui-ci est particuli\`erement laid.}
\end{table}
\section{Introduction}
\subsection{Paragraphes introductifs}
Des listes~:
\begin{enumerate}\tightlist
\item voici le premier point,
\item voici le deuxi\`eme point,
\item voici le troisi\`eme point,
\end{enumerate}
pour une liste ramass\'ee, ou
\begin{enumerate}
\item voici le premier point,
\item voici le deuxi\`eme point,
\item voici le troisi\`eme point,
\end{enumerate}
pour une liste a\'er\'ee.

\chapter[Ce qui suit][\`A suivre]{La suite} 
% dans la table des matieres - en entete - en titre
\chapterprecishere{Ceci n'est que du copier-coller de pages de Wikipedia
  pour tester la mise en page}
Maurice Couette\marginparnote{Ceci est une note, mais il y en aura
  d'autres.\label{note1}}, n\'e \`a Tours\marginparnote{Tr\`es
  belle ville de province, le long de la Loire avec beaucoup de
  chateaux.} le 9 janvier 1858 et d\'ec\'ed\'e le 18 ao�t 1943, est un
physicien~\cite{Albano_1996_PRL} fran\c{c}ais dont les travaux
port\`erent principalement sur la m\'ecanique des fluides et
particuli\`erement sur la rh\'eologie. Son nom est principalement
associ\'e \`a l'\'ecoulement de Couette. On lui doit \'egalement un
viscosim\`etre \`a cylindres concentriques qui porte son
nom~\cite{Aldana_2007_PRL}.
Et maintenant un autre paragraphe
\section{Et apr\`es~?}

Une autre section.
\begin{equation}
E = \frac{1}{2}kx^2
\end{equation}
\subsection{Tout autre chose}
\subsubsection{Qu'y aurait-il apr\`es~?}
\begin{verse}
Demain, d\`es l'aube, \`a l'heure o\`u blanchit la campagne,\\
Je partirai. Vois-tu, je sais que tu m'attends.\\
J'irai par la for\^et, j'irai par la montagne.\\
Je ne puis demeurer loin de toi plus longtemps.\marginparnote{Notez
  l'illustration dans la marge, tr\`es bien choisie. Ce n'est un
  \emph{flottant} au sens de \LaTeX, mais sa position n'est pas
  absol\^ument fix\'ee non plus. Ici sans
  l\'egende, voir plus loin avec une l\'egende.}\\[\baselineskip]

\figinmargin{sketch.eps}

Je marcherai les yeux fix\'es sur mes pens\'ees,\\
Sans rien voir au dehors, sans entendre aucun bruit,\\
Seul, inconnu, le dos courb\'e, les mains crois\'ees,\\
Triste, et le jour pour moi sera comme la nuit.\\[\baselineskip]

Je ne regarderai ni l'or du soir qui tombe,\\
Ni les voiles au loin descendant vers Harfleur,\\
Et quand j'arriverai, je mettrai sur ta tombe\\
Un bouquet de houx vert et de bruy\`ere en fleur.\\
\end{verse}

Victor-Marie Hugo, n\'e le 26 f\'evrier 1802 \`a Besan\c{c}on et mort
le 22 mai 1885 \`a Paris, est un \'ecrivain, dramaturge, po\`ete,
homme politique, acad\'emicien et intellectuel engag\'e fran\c{c}ais,
consid\'er\'e comme le plus important des \'ecrivains romantiques de
langue fran\c{c}aise et un des plus importants \'ecrivains de la
litt\'erature fran\c{c}aise.

Victor Hugo occupe une place exceptionnelle dans l'histoire des
lettres fran\c{c}aises et domine le dix-neuvi\`eme si\`ecle par la
diversit\'e, l'ampleur et la dur\'ee de ses cr\'eations
litt\'eraires. Il est en effet po\`ete lyrique avec des recueils comme
Odes et Ballades (1826), les Feuilles d'automne (1832) ou les
Contemplations (1856), c\'el\`ebres pour l'\'evocation de sa fille
L\'eopoldine morte, mais il est aussi po\`ete engag\'e contre
Napol\'eon III dans Les Ch\^atiments (1853) ou encore po\`ete \'epique
avec La L\'egende des si\`ecles (1859 et 1877). Il est en m\^eme temps
un formidable romancier du peuple, avec par exemple Notre-Dame de
Paris (1831) ou les Mis\'erables (1862), et un th\'eoricien du drame
romantique qu'il illustre pendant une d\'ecennie avec principalement
Hernani en 1830 et Ruy Blas en 1838. 

Son \oe{}uvre monumentale comporte \'egalement des discours politiques \`a
la Chambre des pairs, par exemple sur la peine de mort, sur l'\'ecole
ou sur l'Europe, des r\'ecits de voyages (Le Rhin, 1842 ou Choses vues
, posthumes, 1887 et 1890), et une correspondance abondante. 

Victor Hugo a fortement contribu\'e au renouvellement de la po\'esie
et du th\'e\^atre en tant que chef de file du mouvement romantique ; il
a \'et\'e admir\'e par ses contemporains et l'est encore m\^eme si il
a \'et\'e aussi contest\'e par certains auteurs modernes pour les
surabondances pr\'esentes dans ses textes. Il a aussi permis \`a de
nombreuses g\'en\'erations de d\'evelopper une r\'eflexion sur
l'engagement de l'\'ecrivain dans la vie politique et sociale gr\^ace
\`a ses multiples prises de position qui le condamneront \`a l'exil
pendant les vingt ans du Second Empire. 

\figinmargin[\smallcaption{Premi\`ere illustration}{Illustration de
  marge avec l\'egende et titre\label{illustration1}}]{sketch.eps}

Ses choix, \`a la fois moraux et politiques, de la deuxi\`eme partie
de sa vie et son \oe{}uvre hors du commun ont fait de lui un personnage
embl\'ematique que la Troisi\`eme R\'epublique a honor\'e \`a sa mort
le 22 mai 1885 par des fun\'erailles nationales grandioses qui ont
accompagn\'e le transfert de sa d\'epouille au Panth\'eon, le 31 mai
1885. 
\begin{figure}
  \centering
  \includegraphics[width=\textwidth]{sketch}
  \caption[Figure simple]{Figure de taille normale dans le texte, et
    pleins d'explications pour comprendre les arcanes d'un sch\'ema
    de bifurcation%
\label{figure:fignormal}}
\end{figure}

Victor-Marie Hugo\marginparnote{J'ai eu une baisse d'inventivit\'e,
  \emph{bis repetita...}}, n\'e le 26 f\'evrier 1802 \`a Besan\c{c}on et mort le 22 mai
1885 \`a Paris, est un \'ecrivain, dramaturge, po\`ete, homme politique,
acad\'emicien et intellectuel engag\'e fran\c{c}ais, consid\'er\'e comme le plus
important des \'ecrivains romantiques de langue fran\c{c}aise et un des plus
importants \'ecrivains de la litt\'erature fran\c{c}aise. 

\figinmargin[\smallcaption{}{Illustration sans titre\label{ill2}}]{sketch.eps}

Victor Hugo occupe une place exceptionnelle dans l'histoire des
lettres fran\c{c}aises et domine le dix-neuvi\`eme si\`ecle par la
diversit\'e, l'ampleur et la dur\'ee de ses cr\'eations
litt\'eraires. Il est en effet po\`ete lyrique avec des recueils comme
Odes et Ballades (1826), les Feuilles d'automne (1832) ou les
Contemplations (1856), c\'el\`ebres pour l'\'evocation de sa fille
L\'eopoldine morte, mais il est aussi po\`ete engag\'e contre
Napol\'eon III dans Les Ch\^atiments (1853) ou encore po\`ete \'epique
avec La L\'egende des si\`ecles (1859 et 1877). Il est en m\^eme temps
un formidable romancier du peuple, avec par exemple Notre-Dame de
Paris (1831) ou les Mis\'erables (1862), et un th\'eoricien du drame
romantique qu'il illustre pendant une d\'ecennie avec principalement
Hernani en 1830 et Ruy Blas en 1838. 

Son \oe{}uvre monumentale comporte \'egalement des discours politiques \`a
la Chambre des pairs, par exemple sur la peine de mort, sur l'\'ecole
ou sur l'Europe, des r\'ecits de voyages (Le Rhin, 1842 ou Choses vues
, posthumes, 1887 et 1890), et une correspondance abondante. 

Victor Hugo a fortement contribu\'e au renouvellement de la po\'esie
et du th\'e\^atre en tant que chef de file du mouvement romantique ; il
a \'et\'e admir\'e par ses contemporains et l'est encore m\^eme si il
a \'et\'e aussi contest\'e par certains auteurs modernes pour les
surabondances pr\'esentes dans ses textes. Il a aussi permis \`a de
nombreuses g\'en\'erations de d\'evelopper une r\'eflexion sur
l'engagement de l'\'ecrivain dans la vie politique et sociale gr\^ace
\`a ses multiples prises de position qui le condamneront \`a l'exil
pendant les vingt ans du Second Empire. 

Ses choix, \`a la fois moraux et politiques, de la deuxi\`eme partie
de sa vie et son \oe{}uvre hors du commun ont fait de lui un personnage
embl\'ematique que la Troisi\`eme R\'epublique a honor\'e \`a sa mort
le 22 mai 1885 par des fun\'erailles nationales grandioses qui ont
accompagn\'e le transfert de sa d\'epouille au
Panth\'eon, le 31 mai 1885.


Victor-Marie Hugo, n\'e le 26 f\'evrier 1802 \`a Besan\c{c}on et mort
le 22 mai 1885 \`a Paris, est un \'ecrivain, dramaturge, po\`ete,
homme politique, acad\'emicien et intellectuel engag\'e fran\c{c}ais,
consid\'er\'e comme le plus important des \'ecrivains romantiques de
langue fran\c{c}aise et un des plus importants \'ecrivains de la
litt\'erature fran\c{c}aise.

Victor Hugo occupe une place exceptionnelle dans l'histoire des
lettres fran\c{c}aises et domine le dix-neuvi\`eme si\`ecle par la
diversit\'e, l'ampleur et la dur\'ee de ses cr\'eations
litt\'eraires. Il est en effet po\`ete lyrique avec des recueils comme
Odes et Ballades (1826), les Feuilles d'automne (1832) ou les
Contemplations (1856), c\'el\`ebres pour l'\'evocation de sa fille
L\'eopoldine morte, mais il est aussi po\`ete engag\'e contre
Napol\'eon III dans Les Ch\^atiments (1853) ou encore po\`ete \'epique
avec La L\'egende des si\`ecles (1859 et 1877). Il est en m\^eme temps
un formidable romancier du peuple, avec par exemple Notre-Dame de
Paris (1831) ou les Mis\'erables (1862), et un th\'eoricien du drame
romantique qu'il illustre pendant une d\'ecennie avec principalement
Hernani en 1830 et Ruy Blas en 1838. 

Son oeuvre monumentale comporte \'egalement des discours politiques \`a
la Chambre des pairs, par exemple sur la peine de mort, sur l'\'ecole
ou sur l'Europe, des r\'ecits de voyages (Le Rhin, 1842 ou Choses vues
, posthumes, 1887 et 1890), et une correspondance abondante. 

Victor Hugo a fortement contribu\'e au renouvellement de la po\'esie
et du th\'e\^atre en tant que chef de file du mouvement romantique ; il
a \'et\'e admir\'e par ses contemporains et l'est encore m\^eme si il
a \'et\'e aussi contest\'e par certains auteurs modernes pour les
surabondances pr\'esentes dans ses textes. Il a aussi permis \`a de
nombreuses g\'en\'erations de d\'evelopper une r\'eflexion sur
l'engagement de l'\'ecrivain dans la vie politique et sociale gr\^ace
\`a ses multiples prises de position qui le condamneront \`a l'exil
pendant les vingt ans du Second Empire. 
\figinmargin[\smallcaption{Encore une ?}{juste pour le
  plaisir}]{sketch.eps}

Ses choix, \`a la fois moraux et politiques, de la deuxi\`eme partie
de sa vie et son oeuvre hors du commun ont fait de lui un personnage
embl\'ematique que la Troisi\`eme R\'epublique a honor\'e \`a sa mort
le 22 mai 1885 par des fun\'erailles nationales grandioses qui ont
accompagn\'e le transfert de sa d\'epouille au Panth\'eon, le 31 mai
1885. 

\begin{bigfigure}
  \largegraphic[\bigfigwidth]{sketch}
  \caption[Une grande figure]{Une figure de taille maximale, et
    pleins d'explications pour comprendre les arcanes d'un sch\'ema
    de bifurcation \label{bigfig}}
\end{bigfigure}

\begin{verse}
Demain, d\`es l'aube, \`a l'heure o\`u blanchit la campagne,\\
Je partirai. Vois-tu, je sais que tu m'attends.\\
J'irai par la for\^et, j'irai par la montagne.\\
Je ne puis demeurer loin de toi plus longtemps.\\[\baselineskip]

Je marcherai les yeux fix\'es sur mes pens\'ees,\\
Sans rien voir au dehors, sans entendre aucun bruit,\\
Seul, inconnu, le dos courb\'e, les mains crois\'ees,\\
Triste, et le jour pour moi sera comme la nuit.\\[\baselineskip]

Je ne regarderai ni l'or du soir qui tombe,\\
Ni les voiles au loin descendant vers Harfleur,\\
Et quand j'arriverai, je mettrai sur ta tombe\\
Un bouquet de houx vert et de bruy\`ere en fleur.\\
\end{verse}

Victor-Marie Hugo, n\'e le 26 f\'evrier 1802 \`a Besan\c{c}on et mort
le 22 mai 1885 \`a Paris, est un \'ecrivain, dramaturge, po\`ete,
homme politique, acad\'emicien et intellectuel engag\'e fran\c{c}ais,
consid\'er\'e comme le plus important des \'ecrivains romantiques de
langue fran\c{c}aise et un des plus importants \'ecrivains de la
litt\'erature fran\c{c}aise.

Victor Hugo occupe une place exceptionnelle dans l'histoire des
lettres fran\c{c}aises et domine le dix-neuvi\`eme si\`ecle par la
diversit\'e, l'ampleur et la dur\'ee de ses cr\'eations
litt\'eraires. Il est en effet po\`ete lyrique avec des recueils comme
Odes et Ballades (1826), les Feuilles d'automne (1832) ou les
Contemplations (1856), c\'el\`ebres pour l'\'evocation de sa fille
L\'eopoldine morte, mais il est aussi po\`ete engag\'e contre
Napol\'eon III dans Les Ch\^atiments (1853) ou encore po\`ete \'epique
avec La L\'egende des si\`ecles (1859 et 1877). Il est en m\^eme temps
un formidable romancier du peuple, avec par exemple Notre-Dame de
Paris (1831) ou les Mis\'erables (1862), et un th\'eoricien du drame
romantique qu'il illustre pendant une d\'ecennie avec principalement
Hernani en 1830 et Ruy Blas en 1838. 

Son \oe{}uvre monumentale comporte \'egalement des discours politiques \`a
la Chambre des pairs, par exemple sur la peine de mort, sur l'\'ecole
ou sur l'Europe, des r\'ecits de voyages (Le Rhin, 1842 ou Choses vues
, posthumes, 1887 et 1890), et une correspondance abondante. 

Victor Hugo a fortement contribu\'e au renouvellement de la po\'esie
et du th\'e\^atre en tant que chef de file du mouvement romantique ; il
a \'et\'e admir\'e par ses contemporains et l'est encore m\^eme si il
a \'et\'e aussi contest\'e par certains auteurs modernes pour les
surabondances pr\'esentes dans ses textes. Il a aussi permis \`a de
nombreuses g\'en\'erations de d\'evelopper une r\'eflexion sur
l'engagement de l'\'ecrivain dans la vie politique et sociale gr\^ace
\`a ses multiples prises de position qui le condamneront \`a l'exil
pendant les vingt ans du Second Empire. 

Ses choix, \`a la fois moraux et politiques, de la deuxi\`eme partie
de sa vie et son \oe{}uvre hors du commun ont fait de lui un personnage
embl\'ematique que la Troisi\`eme R\'epublique a honor\'e \`a sa mort
le 22 mai 1885 par des fun\'erailles nationales grandioses qui ont
accompagn\'e le transfert de sa d\'epouille au Panth\'eon, le 31 mai
1885. 

Victor-Marie Hugo, n\'e le 26 f\'evrier 1802 \`a Besan\c{c}on et mort
le 22 mai 1885 \`a Paris, est un \'ecrivain, dramaturge, po\`ete,
homme politique, acad\'emicien et intellectuel engag\'e fran\c{c}ais,
consid\'er\'e comme le plus important des \'ecrivains romantiques de
langue fran\c{c}aise et un des plus importants \'ecrivains de la
litt\'erature fran\c{c}aise.

Victor Hugo occupe une place exceptionnelle dans l'histoire des
lettres fran\c{c}aises et domine le dix-neuvi\`eme si\`ecle par la
diversit\'e, l'ampleur et la dur\'ee de ses cr\'eations
litt\'eraires. Il est en effet po\`ete lyrique avec des recueils comme
Odes et Ballades (1826), les Feuilles d'automne (1832) ou les
Contemplations (1856), c\'el\`ebres pour l'\'evocation de sa fille
L\'eopoldine morte, mais il est aussi po\`ete engag\'e contre
Napol\'eon III dans Les Ch\^atiments (1853) ou encore po\`ete \'epique
avec La L\'egende des si\`ecles (1859 et 1877). Il est en m\^eme temps
un formidable romancier du peuple, avec par exemple Notre-Dame de
Paris (1831) ou les Mis\'erables (1862), et un th\'eoricien du drame
romantique qu'il illustre pendant une d\'ecennie avec principalement
Hernani en 1830 et Ruy Blas en 1838. 

Son \oe{}uvre monumentale comporte \'egalement des discours politiques \`a
la Chambre des pairs, par exemple sur la peine de mort, sur l'\'ecole
ou sur l'Europe, des r\'ecits de voyages (Le Rhin, 1842 ou Choses vues
, posthumes, 1887 et 1890), et une correspondance abondante. 

Victor Hugo a fortement contribu\'e au renouvellement de la po\'esie
et du th\'e\^atre en tant que chef de file du mouvement romantique ; il
a \'et\'e admir\'e par ses contemporains et l'est encore m\^eme si il
a \'et\'e aussi contest\'e par certains auteurs modernes pour les
surabondances pr\'esentes dans ses textes. Il a aussi permis \`a de
nombreuses g\'en\'erations de d\'evelopper une r\'eflexion sur
l'engagement de l'\'ecrivain dans la vie politique et sociale gr\^ace
\`a ses multiples prises de position qui le condamneront \`a l'exil
pendant les vingt ans du Second Empire. 

Ses choix, \`a la fois moraux et politiques, de la deuxi\`eme partie
de sa vie et son \oe{}uvre hors du commun ont fait de lui un personnage
embl\'ematique que la Troisi\`eme R\'epublique a honor\'e \`a sa mort
le 22 mai 1885 par des fun\'erailles nationales grandioses qui ont
accompagn\'e le transfert de sa d\'epouille au Panth\'eon, le 31 mai
1885. 
Victor-Marie Hugo, n\'e le 26 f\'evrier 1802 \`a Besan\c{c}on et mort
le 22 mai 1885 \`a Paris, est un \'ecrivain, dramaturge, po\`ete,
homme politique, acad\'emicien et intellectuel engag\'e fran\c{c}ais,
consid\'er\'e comme le plus important des \'ecrivains romantiques de
langue fran\c{c}aise et un des plus importants \'ecrivains de la
litt\'erature fran\c{c}aise.

Victor Hugo occupe une place exceptionnelle dans l'histoire des
lettres fran\c{c}aises et domine le dix-neuvi\`eme si\`ecle par la
diversit\'e, l'ampleur et la dur\'ee de ses cr\'eations
litt\'eraires. Il est en effet po\`ete lyrique avec des recueils comme
Odes et Ballades (1826), les Feuilles d'automne (1832) ou les
Contemplations (1856), c\'el\`ebres pour l'\'evocation de sa fille
L\'eopoldine morte, mais il est aussi po\`ete engag\'e contre
Napol\'eon III dans Les Ch\^atiments (1853) ou encore po\`ete \'epique
avec La L\'egende des si\`ecles (1859 et 1877). Il est en m\^eme temps
un formidable romancier du peuple, avec par exemple Notre-Dame de
Paris (1831) ou les Mis\'erables (1862), et un th\'eoricien du drame
romantique qu'il illustre pendant une d\'ecennie avec principalement
Hernani en 1830 et Ruy Blas en 1838. 

Son \oe{}uvre monumentale comporte \'egalement des discours politiques \`a
la Chambre des pairs, par exemple sur la peine de mort, sur l'\'ecole
ou sur l'Europe, des r\'ecits de voyages (Le Rhin, 1842 ou Choses vues
, posthumes, 1887 et 1890), et une correspondance abondante. 

Victor Hugo a fortement contribu\'e au renouvellement de la po\'esie
et du th\'e\^atre en tant que chef de file du mouvement romantique ; il
a \'et\'e admir\'e par ses contemporains et l'est encore m\^eme si il
a \'et\'e aussi contest\'e par certains auteurs modernes pour les
surabondances pr\'esentes dans ses textes. Il a aussi permis \`a de
nombreuses g\'en\'erations de d\'evelopper une r\'eflexion sur
l'engagement de l'\'ecrivain dans la vie politique et sociale gr\^ace
\`a ses multiples prises de position qui le condamneront \`a l'exil
pendant les vingt ans du Second Empire. 

Ses choix, \`a la fois moraux et politiques, de la deuxi\`eme partie
de sa vie et son \oe{}uvre hors du commun ont fait de lui un personnage
embl\'ematique que la Troisi\`eme R\'epublique a honor\'e \`a sa mort
le 22 mai 1885 par des fun\'erailles nationales grandioses qui ont
accompagn\'e le transfert de sa d\'epouille au Panth\'eon, le 31 mai
1885. 
\part{Une autre pour tester la num\'erotation}
\appendix
\appendixpage
\chapter{La distribution de Weibull}
\section{Math\'ematique}
\section{Physique}
\subsection{Physique statistique}
\backmatter
% Tables de figures
\listoffigures
\listofillustrations
\cleardoublepage
% Liste des tableaux
\listoftables

\bibliography{/home/guillaume/mybib}
\bibliographystyle{alpha}
\end{document}
